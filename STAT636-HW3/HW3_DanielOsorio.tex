\documentclass[12pt,a4paper]{paper}
\usepackage[utf8]{inputenc}
\usepackage[english]{babel}
\usepackage{amsmath}
\usepackage{enumitem}
\usepackage{amsfonts}
\usepackage{amssymb}
\usepackage[left=1cm,right=1cm,top=1.5cm,bottom=2cm]{geometry}
\usepackage{Sweave}
\begin{document}
\title{STAT636 - Homework 3\\\small{Daniel Osorio - dcosorioh@tamu.edu\\Department of Veterinary Integrative Biosciences\\Texas A\&M University}}
\maketitle
\Sconcordance{concordance:HW3_DanielOsorio.tex:HW3_DanielOsorio.Rnw:%
1 8 1 1 0 11 1 1 28 27 0 1 1 9 0 1 2 1 1}

\begin{enumerate}
\item Consider the Auto data. These represent a two-factor experiment on cars. The two factors are (i) the number of cylinders (4 and 6 were considered) and (ii) origin (three origins were considered). We have 4 cars under each of the $2 \times 3 = 6$ factor combinations. For each car, we have measurements of three weight variables: $X_{1}$ = displacement, $X_{2}$ = horsepower, and $X_{3}$ = acceleration. So, in terms of a two-way MANOVA model, $g = 2$, $b = 3$, and $n = 4$.
\begin{enumerate}
\item Test for a location effect, a variety effect, and a location-variety interaction at $\alpha = 0.05$. Do this using the manova function in R. Overall, what do you conclude about these data?
\item Construct the two-way MANOVA table by computing SSPFAC 1, SSPFAC 2, SSPINT, SSPRES, and SSPCOR. Provide R code that matches the Wilks’ statistics computed by manova. Note that your p-values (computed according to the notes) will not match those of manova, because the distributional results we have learned for two-way MANOVA are large-sample approximations. That said, how do your p-values compare to those of manova?
\end{enumerate}
\item Conduct a simulation study to investigate the coverage probabilities of different confidence
interval types with multivariate data. Let the sample size be $n = 30$, the number of variables $p = 4$, the number of simulations $B = 10000$, and \[\mu' =\left(0, 0, 0, 0, 0\right)\] and \[\Sigma = \left(\begin{array}{ccccc}1.0&0.6&0.6&0.6&0.6\\0.6&1.0&0.6&0.6&0.6\\0.6&0.6&1.0&0.6&0.6\\0.6&0.6&0.6&1.0&0.6\\0.6&0.6&0.6&0.6&1.0\end{array}\right)\] For each of $B$ times, simulate a dataset of size $n$ from the $N_{p}\left(\mu,\Sigma\right)$ distribution, compute 95\% confidence intervals of types one-at-a-time, $T^{2}$ simultaneous, and Bonferroni simultaneous, and record whether each interval contains its corresponding population mean component value. Report a $3 \times 5$ matrix of estimated coverage probabilities. The rows of your matrix should correspond to the 3 different interval types, and the columns should correspond to the p mean components; be sure to clearly indicate which row goes with which interval type. Comment on the performance of the different interval types.
\end{enumerate}
\end{document}

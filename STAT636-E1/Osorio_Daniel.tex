\documentclass[12pt,a4paper]{paper}
\usepackage[utf8]{inputenc}
\usepackage[english]{babel}
\usepackage{amsmath}
\usepackage{enumitem}
\usepackage{amsfonts}
\usepackage{arydshln}
\usepackage{amssymb}
\usepackage{multicol}
\usepackage[left=1cm,right=1cm,top=1.5cm,bottom=2cm]{geometry}
\usepackage{Sweave}
\begin{document}
\title{STAT636 - Exam 1\\\small{Daniel Osorio - dcosorioh@tamu.edu\\Department of Veterinary Integrative Biosciences\\Texas A\&M University}}
\maketitle
\Sconcordance{concordance:Osorio_Daniel.tex:Osorio_Daniel.Rnw:%
1 10 1 1 0 5 1 1 6 14 0 1 2 1 1 1 3 8 0 2 2 1 0 1 2 6 0 1 2 7 0 2 2 7 0 %
2 2 7 0 2 2 10 0 2 2 1 0 1 1 5 0 1 1 6 0 1 2 1 1 1 2 4 0 1 2 1 1 1 2 1 %
0 1 1 1 2 7 0 2 1 1 2 9 0 1 1 1 2 6 0 1 4 11 0 1 2 1 1 1 2 1 0 1 1 6 0 %
1 2 1 5 12 0 1 2 1 1 1 2 1 0 1 2 7 0 1 2 7 0 1 1 7 0 1 2 2 1}

\begin{enumerate}
\item 
\begin{Schunk}
\begin{Sinput}
> print(Sigma <- matrix(c(1,0.125,0.2,
+                         0.125,0.25,0.1,
+                         0.2,0.1,0.64), 
+                       ncol = 3, 
+                       byrow = TRUE))
\end{Sinput}
\begin{Soutput}
      [,1]  [,2] [,3]
[1,] 1.000 0.125 0.20
[2,] 0.125 0.250 0.10
[3,] 0.200 0.100 0.64
\end{Soutput}
\end{Schunk}
\begin{enumerate}
\item Correlation Value 
\begin{Schunk}
\begin{Sinput}
> # cov(x,y)/ sd(x) * sd(y)
> Sigma[2,3] / (sqrt(Sigma[2,2]) * sqrt(Sigma[3,3]))
\end{Sinput}
\begin{Soutput}
[1] 0.25
\end{Soutput}
\end{Schunk}
\item First eigen value and eigen vector of sigma
\begin{Schunk}
\begin{Sinput}
> eigenDSigma <- eigen(Sigma)
> # First eigen value of Sigma
> eigenDSigma$values[1]
\end{Sinput}
\begin{Soutput}
[1] 1.116835
\end{Soutput}
\begin{Sinput}
> # First eigen vector of Sigma
> eigenDSigma$vectors[,1]
\end{Sinput}
\begin{Soutput}
[1] 0.8939588 0.1764355 0.4119565
\end{Soutput}
\end{Schunk}
\item Determinant of sigma
\begin{Schunk}
\begin{Sinput}
> det(Sigma)
\end{Sinput}
\begin{Soutput}
[1] 0.135
\end{Soutput}
\end{Schunk}
\item Is Sigma PD? \textit{Yes, because all of their eigen values are positive}
\begin{Schunk}
\begin{Sinput}
> all(eigenDSigma$values > 0)
\end{Sinput}
\begin{Soutput}
[1] TRUE
\end{Soutput}
\end{Schunk}
\item What is the inverse of $\Sigma$?
\begin{Schunk}
\begin{Sinput}
> solve(Sigma)
\end{Sinput}
\begin{Soutput}
           [,1]       [,2]       [,3]
[1,]  1.1111111 -0.4444444 -0.2777778
[2,] -0.4444444  4.4444444 -0.5555556
[3,] -0.2777778 -0.5555556  1.7361111
\end{Soutput}
\end{Schunk}
\item What is the first eigenvalue and eigenvector of $\Sigma^{-1}$
\begin{Schunk}
\begin{Sinput}
> eigenDinvSigma <- eigen(solve(Sigma))
> eigenDinvSigma$values[1]
\end{Sinput}
\begin{Soutput}
[1] 4.59644
\end{Soutput}
\begin{Sinput}
> eigenDinvSigma$vectors[1,]
\end{Sinput}
\begin{Soutput}
[1] 0.1103844 0.4343420 0.8939588
\end{Soutput}
\end{Schunk}
\end{enumerate}
\item
\begin{Schunk}
\begin{Sinput}
> X <- read.csv("exam_1.csv")
\end{Sinput}
\end{Schunk}
\begin{enumerate}
\item 90 \% CI region 
\begin{Schunk}
\begin{Sinput}
> n <- nrow(X)
> p <- ncol(X)
> # Center
> print(x_bar <- colMeans(X))
\end{Sinput}
\begin{Soutput}
        V1         V2         V3 
0.12691538 0.07514347 0.16364740 
\end{Soutput}
\begin{Sinput}
> S <- var(X)
> eigen_S <- eigen(S)
> # Primary axes
> eigen_S$vectors
\end{Sinput}
\begin{Soutput}
           [,1]       [,2]        [,3]
[1,] -0.4667773  0.8133957  0.34714044
[2,] -0.2391548  0.2618053 -0.93502029
[3,] -0.8514247 -0.5194665  0.07232275
\end{Soutput}
\begin{Sinput}
> c2 <- (((n - 1) * p) / (n - p)) * qf(0.90, p, n - p)
> # Half lengths
> sqrt(eigen_S$values / n) * sqrt(c2)
\end{Sinput}
\begin{Soutput}
[1] 0.6070969 0.2042265 0.1092275
\end{Soutput}
\begin{Sinput}
> # 90% Confidence region
> print(CR90 <- sapply(seq_len(p), function(x){
+   x_bar[x] + c(-1, 1) * sqrt(c2 * S[x, x] / n)
+ }))
\end{Sinput}
\begin{Soutput}
           [,1]       [,2]       [,3]
[1,] -0.2037449 -0.1102466 -0.3640836
[2,]  0.4575757  0.2605336  0.6913784
\end{Soutput}
\end{Schunk}
\item Interpretation:\textit{A 90\% confidence region is a range of values that you can be certain contains the 90\% of the population values.}
\item Is the given value contained? \textit{No.}
\begin{Schunk}
\begin{Sinput}
> mu_0 <- c(-0.5, 0, 0.5)
> all(CR90[1,] < mu_0 & CR90[2,] > mu_0)
\end{Sinput}
\begin{Soutput}
[1] FALSE
\end{Soutput}
\end{Schunk}
\item 90\% Bonferroni simultaneous confidence intervals
\begin{Schunk}
\begin{Sinput}
> # Bonferroni 
> sapply(seq_len(p), function(x){
+ x_bar[1] + c(-1, 1) * qt(1 - 0.1 / (2 * p), n - 1) * sqrt(S[1, 1] / n)
+ })
\end{Sinput}
\begin{Soutput}
           [,1]       [,2]       [,3]
[1,] -0.1488520 -0.1488520 -0.1488520
[2,]  0.4026828  0.4026828  0.4026828
\end{Soutput}
\end{Schunk}
\item Interpretation:\textit{A 90\% confidence interval is a range of values that you can be 90\% certain contains the true mean of the population.}
\item
\begin{Schunk}
\begin{Sinput}
> S_0 <- (t(X) - mu_0) %*% t(t(X) - mu_0) / (n - 1)
> # T2 Statistic Value
> print(T_2_stat <- n * t(x_bar - mu_0) %*% solve(S) %*% (x_bar - mu_0))
\end{Sinput}
\begin{Soutput}
         [,1]
[1,] 90.75469
\end{Soutput}
\begin{Sinput}
> # P - Value
> print(p_value <- 1 - pf((n - p) * T_2_stat / ((n - 1) * p), p, n - p))
\end{Sinput}
\begin{Soutput}
            [,1]
[1,] 9.10505e-11
\end{Soutput}
\begin{Sinput}
> T_2_stat <= c2
\end{Sinput}
\begin{Soutput}
      [,1]
[1,] FALSE
\end{Soutput}
\end{Schunk}
\end{enumerate}
\end{enumerate}
\end{document}
